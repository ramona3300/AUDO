\section{Organisation}
\label{cha:Organisation}
In diesem Abschnitt wird das Projektmanagement des Teams beschrieben, das zum erfolgreichen Abschluss des Projektes beigetragen hat. Bei einem Team mit fünf Personen ist es wichtig Maßnahmen zur Dateiverwaltung und Organisation zu treffen, um eine gute Zusammenarbeit zu gewährleisten.

\subsection{Gruppentreffen und Organisation}
\label{sec:gruppentreffenundorganisation}
\tod{oder eher paragraph hier als fette Überschriften?}
\paragraph{Gruppentreffen}
Bei unserem ersten Teamtreffen entschieden wir uns für eine Versionsverwaltung, sowie eine Organisationsplattform und richteten diese zusammen ein.

Im zwei Wochen Takt traf sich vorzugsweise das gesamte Team mit einem Betreuer des Projektes. Mit ihm besprachen wir den aktuellen Stand unseres Projektes und gegebenenfalls über Probleme im Team. Diese Treffen waren organisatorischen Zwecken gewidmet, es wurden keine Fragen direkt zur Implementierung geklärt. Als das Projekt grundlegend funktionierte, wurden Anregungen zur Erweiterung der Aufgabenstellung gegeben. 

Zur Besprechung von Regelungs- und Implementierungsfragen, gab es annähernd jede zweite Woche ein Regelungstechniktreffen, bei dem sich zwei Teammitglieder jeder Gruppe einfanden. Wir tauschten uns untereinander über den Stand der jeweiligen Gruppen aus und disktutierten verschiedene Fragen. Für weitere Fragen und Anregungen stand dort ein Betreuer der Regelungstechnik zur Verfügung.

Zusätzlich zu den Beratungsgesprächen legten wir feste wöchentliche Treffen mit allen Gruppenmitgliedern fest. Jeder berichtete von seinem Vorgehen der vergangenen Woche, so dass alle im Team den Überblick über den aktuellen Stand des Projektes hatten. Daraufhin wurden Ideen und Anregungen zusammengetragen und die Planung für die nächste Woche erarbeitet. Mit Hilfe der im Folgenden noch erwähnten Organisationsplattform Trello wurden die Aufgaben erfasst und den interessierten Gruppenmitgliedern zugeordnet. 

\paragraph{Team}

Wir entschieden die Zeitplanung dynamisch zu halten, da es schwer einschätzbar war, wie viel Zeit eine einzelne Aufgabe in Anspruch nimmt. Erfahrungen aus anderen Projekten haben gezeigt, dass dieses Vorgehen sinnvoll ist. Wir setzten uns als Ziel, dass nach vier von fünf Monaten der Projektzeit eine funktionierende Version vorhanden sein soll, um genügend Zeit für die Optimierung und Dokumentation einzuplanen und gegebenenfalls einen ausreichenden Zeitpuffer zur Verfügung zu haben. Jede Woche wurde bei dem Gruppentreffen festgelegt, wer sich welcher neuen Aufgabe annimmt und wir schätzten erneut ab, wie viel Zeit eine angefangene Aufgabe noch beanspruchen wird.


Nachdem sich alle in die Dokumentation eingelesen hatten, erarbeiteten wir erste Grundlagen zusammen, damit jeder ein Grundwissen über das Modellauto hatte. Es bildeten sich Aufgabengruppen für Organisation, Regelung und Bildverarbeitung, diese wurden weitestgehend parallel behandelt. Die Verantwortung für diese Teilgruppen wurde auf die Mitwirkenden aufgeteilt. Jedes Gruppenmitglied konnte bei jeder Aufgabengruppe helfen, allerdings ist es wichtig Verantwortliche festzulegen, um den Überblick zu bewahren und Zielführend zu agieren. Als die Tests der Bildverarbeitung und Regelung die aufgestellten Bedingungen erfüllten, schlossen wir diese Aufgabenbereiche zusammen und erlangten dadurch das Fahren durch den Rundkurs. Neben der Optimierung des Rundkurses wandte sich eine Subgruppe dem zweiten Thema, Hinderniserkennung mit Spurwechsel, zu.

Es gab viele Aufgaben, die direkt am Fahrzeug getestet werden mussten. Da das gleichzeitige Testen mehrerer Aufgaben meist nicht möglich war, unterstützte man beim Zusammenkommen die anderen Teammitgliedern über die Aufgabenverteilung hinaus. Jeder teilte sich seine Zeit jede Woche selbst ein. Wir kommunizierten, welche Aufgabe wann bearbeitet wird, damit Interessengruppen zusammenfinden konnten. 

\subsection{Aufgabenverwaltung}
\label{sec:aufgabenverwaltung}


\subsection{Versionsverwaltung}
\label{sec:versionsverwaltung}
Angesichts guter Erfahrungen stand schnell fest, dass wir \textbf{Git} als Versionsverwaltung verwenden. Die meisten Teammitgleider verfügten bereits über Kenntnisse mit \textbf{Git} und musssten sich nicht in neue Verwaltungssturkturen einarbeiten. Der Vorteil von Versionsverwaltungen ist es, dass man alte Programmierzustände gesichert sind und man diese wieder herstellen kann. Wir nutzten \textbf{Git} zur Versionierung unseres Quellcodes und stellten dadurch auch das fehlerfreie Programmieren im Team sicher. Ohne Verwaltung des Codes ist es fehleranfällig  mti mehreren Leuten an dem gleichen Code zu programmieren. 
