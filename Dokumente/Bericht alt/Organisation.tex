\chapter{Organisation}
\label{cha:Organisation}
In diesem Abschnitt wird das Projektmanagment des Teams beschrieben, das zum erfolgreichen Abschluss unseres Projektes beigetragen hat. Bei einem Team mit fünf Personen ist es wichtig Maßnahmen zur Dateiverwaltung und Organisation zu treffen, um eine gute Zusammenarbeit anzuregen/anzusteuern.

\section{Gruppentreffen und Organisation}
\label{sec:gruppentreffenundorganisation}
\subsection*{Gruppentreffen}
Wir entschieden uns bei unserem ersten Teamtreffen für eine Versionsverwaltung und eine Organisationsplattform und richteten diese zusammen ein.

Im zwei Wochen Takt traf sich vorzugsweise das gesamte Team mit einem Betreuer des Projektes. Mit ihm besprachen wir den aktuellen Stand unseres Projektes und gegebenfalls Probleme im Team. Er gab uns Diese Treffen waren organisatorischen Zwecken gewittmet, es wurden keine Fragen direkt zur Implementierung geklärt. Zur Besprechung von Regelungs- und Implementierungsfragen, gab es annähernd jede zweite Wochen ein Regelungstechniktreffen, bei dem zwei Teammitglieder jeder Gruppe sich trafen. Es wurde der Stand der jeweiligen Gruppen untereinander ausgetauscht und Fragen diskutiert. Für weitere Fragen und Anregungen stand dort ein Betreuer der Regelungstechnik zur Verfügung.

Zusätzlich zu den Beratungsgesprächen legten wir feste wöchentliche Treffen mit allen Gruppenmitgliedern fest. Jeder berichtete von seinem Vorgehen der vergangenen Woche, so dass alle im Team den Überblick über den aktuellen Stand des Projektes hatten. Daraufhin wurden Ideen und Anregungen zusammengetragen und die Planung für die nächste Woche erarbeitet. Mit Hilfe der im Folgenden noch erwähnten Organisationsplattform wurden die Aufgaben erfasst und den interessierten Gruppenmitgliedern zugeordnet. 

\subsection*{Team}



\section{Aufgabenverwaltung}
\label{sec:aufgabenverwaltung}


\section{Versionsverwaltung}
\label{sec:versionsverwaltung}
