\chapter{Organisation}
\label{cha:Organisation}
In diesem Abschnitt wird das Projektmanagment des Teams beschrieben, das zum erfolgreichen Abschluss unseres Projektes beigetragen hat. Bei einem Team mit fünf Personen ist es wichtig Maßnahmen zur Dateiverwaltung und Organisation zu treffen, um eine gute Zusammenarbeit anzusteuern.

\section{Gruppentreffen und Organisation}
\label{sec:gruppentreffenundorganisation}
\subsection*{Gruppentreffen}
Wir entschieden uns bei unserem ersten Teamtreffen für eine Versionsverwaltung und eine Organisationsplattform und richteten diese zusammen ein.

Im zwei Wochen Takt traf sich vorzugsweise das gesamte Team mit einem Betreuer des Projektes. Mit ihm besprachen wir den aktuellen Stand unseres Projektes und gegebenfalls Probleme im Team. Diese Treffen waren organisatorischen Zwecken gewittmet, es wurden keine Fragen direkt zur Implementierung geklärt. Als das Projekt grundlegend funktionierte, wurden Anregungen zur Erweiterung der Aufgabenstellung gegeben. 

Zur Besprechung von Regelungs- und Implementierungsfragen, gab es annähernd jede zweite Wochen ein Regelungstechniktreffen, bei dem sich zwei Teammitglieder jeder Gruppe einfunden. Es wurde der Stand der jeweiligen Gruppen untereinander ausgetauscht und Fragen diskutiert. Für weitere Fragen und Anregungen stand dort ein Betreuer der Regelungstechnik zur Verfügung.

Zusätzlich zu den Beratungsgesprächen legten wir feste wöchentliche Treffen mit allen Gruppenmitgliedern fest. Jeder berichtete von seinem Vorgehen der vergangenen Woche, so dass alle im Team den Überblick über den aktuellen Stand des Projektes hatten. Daraufhin wurden Ideen und Anregungen zusammengetragen und die Planung für die nächste Woche erarbeitet. Mit Hilfe der im Folgenden noch erwähnten Organisationsplattform Trello wurden die Aufgaben erfasst und den interessierten Gruppenmitgliedern zugeordnet. 

\subsection*{Team}

Wir entschieden die Zeitplanung dinamisch zu halten, da es schwer einschätzbar ist, wie viel Zeit eine einzelne Aufgabe in Anspruch nimmt. Erfahrungen aus anderen Projekten haben gezeigt, dass dieses Vorgehen sinnvoll ist. Wir setzten uns als Ziel, dass nach \nicefrac{2}{3} der Projektzeit eine funktionierende Version vorhanden ist, um genügend Zeit für die Optimierung und Dokumentation einzuplanen und gegebenenfalls einen ausreichenden Zeitpuffer zur Verfügung zu haben. Jede Woche bei dem Gruppentreffen wurde festgelegt, wer sich welcher neuen Aufgabe annimmt und wir schätzten erneut ab, wie viel Zeit eine angefangegne Aufgabe noch benötigen wird.


Nachdem sich alle in die Dokumentation eingelesen hatten, erarbeiteten wir erste Grundlagen zusammen, damit jeder ein Grundwissen über das Modellauto hatte. Es bildeten sich Aufgabengruppen für Organisation, Regelung und Bildverarbeitung, diese wurden weitgehendst parallel behandelt. Die Verantwortung für diese Teilgruppen wurde auf die Mitwirkenden aufgeteilt. Jedes Gruppenmitglied konnte bei jeder Aufgabengruppe helfen, allerdings ist es wichtig Verantwortliche festzulegen, um den Überblick zu bewahren und Zielführend zu agieren. Als die Tests der Bildverarbeitung und Regelung die aufgestellten Bedingungen erfüllten, schlossen wir diese Aufgabenbereiche zusammen und erlangten dadurch das Fahren durch den Rundkurs. Neben der Optimierung des Rundkurses wendete sich eine Subgruppe dem zweiten Thema, Hinderniserkennung mit Spurwechsel, zu.

Es gab viele Aufgaben, die direkt am Fahrzeug getestet werden mussten. Da das gleichzeitige Testen mehrerer Aufgaben meist nicht möglich war, unterstützte man beim Zusammenkommen die anderen Teammitgliedern über die Aufgabenverteilung hinaus. Jeder teilte sich seine Zeit selbstständig jede Woche ein. Wir kommunizierten, welche Aufgabe wann bearbeitet wird, damit Interessengruppen zusammenfinden konnten. 

\section{Aufgabenverwaltung}
\label{sec:aufgabenverwaltung}


\section{Versionsverwaltung}
\label{sec:versionsverwaltung}
